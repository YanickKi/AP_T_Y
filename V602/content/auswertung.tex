\section{Auswertung}
\label{sec:Auswertung}
Jegliche Fehlerrechnung wurde mit der python-Bibliothek uncertainties \cite{uncertainties} absolviert.
Trotz dessen sind die Formeln für die Unsicherheiten in den jeweiligen Abschnitten angegeben.
Allgemeine Rechnungen wurden mit der python-Bibliothek numpy \cite{numpy} automatisiert. 
Die graphischen Untersützungen wurden mit Hilfe der python-Bibliothek matplotlib \cite{matplotlib} erstellt.
\subsection{Überprüfung der Bragg-Bedingung}
Zunächst muss die Bragg-Bedingung überprüft werden. 
Die dazu gemessenen Daten sind in der Tabelle \ref{tab:Bragg} aufgetragen. 
Außerdem sind die Werte in der Abbildung \ref{fig:Bragg} graphisch dargestellt.
\begin{table}
    \centering
    \caption{Gemessene Winkel und Impulse}
    \label{tab:Bragg}
    \begin{tabular} {S[table-format=2.1]  S[table-format=2.0] S[table-format=2.1] S[table-format=3.0]  S[table-format=2.1]  
                     S[table-format=3.1]  S[table-format=2.1] S[table-format=3.0]}
        \toprule
        {$\theta \mathbin{/} \si{\degree}$} & {$N$} & {$\theta \mathbin{/} \si{\degree}$} & {$N$} 
    &   {$\theta \mathbin{/} \si{\degree}$} & {$N$} & {$\theta \mathbin{/} \si{\degree}$} & {$N$}\\
    \midrule
    26.0 & 56 & 27.0 & 105 & 28.0 & 212 & 29.0 & 138 \\
    26.1 & 58 & 27.1 & 119 & 28.1 & 215 & 29.1 & 125 \\
    26.2 & 54 & 27.2 & 125 & 28.2 & 218 & 29.2 & 111 \\
    26.3 & 62 & 27.3 & 141 & 28.3 & 215 & 29.3 & 107 \\
    26.4 & 58 & 27.4 & 154 & 28.4 & 208 & 29.4 & 95  \\
    26.5 & 68 & 27.5 & 157 & 28.5 & 189 & 29.5 & 77  \\
    26.6 & 72 & 27.6 & 166 & 28.6 & 189 & 29.6 & 73  \\
    26.7 & 83 & 27.7 & 180 & 28.7 & 176 & 29.7 & 58  \\
    26.8 & 89 & 27.8 & 188 & 28.8 & 164 & 29.8 & 56  \\
    26.9 & 95 & 27.9 & 211 & 28.9 & 149 & 29.9 & 53  \\
    {-}  & {-}& {-}  & {-} & {-}  & {-} & 30.0 & 53  \\
    \bottomrule
    \end{tabular}
\end{table}
\begin{figure}
    \centering
    \caption{Bragg-Bedingung}
    \label{fig:Bragg}
    \includegraphics{build/Bragg.pdf}
\end{figure}
Das Maximum der Impulse $N_\text{max} = \num{218}$ ist bei einem Winkel von $\theta_\text{max} = \ang{28.2;;}$ gemessen worden.
Dieser Winkel weicht vom Sollwinkel $\theta_\text{soll} = \ang{28;;}$ nur um $\ang{0.2;;}$ ab, so dass die Bragg-Bedingung erfüllt ist.
\subsection{Analyse eines Emissionsspektrums der Kupfer-Röntgenröhre}
\begin{figure}
    \centering
    \caption{Das Emissionsspektrum der Kupfer-Röntgenröhre}
    \label{fig:Kupfer}
    \includegraphics{build/Kupfer.pdf}
\end{figure}
Die $K_\alpha$ Linie befindet sich bei  $\theta = \ang{22.5;;}$m, während die $K_\beta$ bei  $\theta = \ang{20.2;;}$ wiederzufinden ist.
Die Halbwertsbreiten betragen
\begin{align*}
    \symup{\Delta}K_\alpha  &= \ang{0.6;;} \\
    \symup{\Delta}K_\beta   &= \ang{0.6;;}
\end{align*}
Mit Hilfe der Bragg Bedingung $REFERENZ$ lässt sich die Wellenlänge bestimmt, wonach die Energie bestimmt werden kann.
Die Energien ergeben sich zu 
\begin{align*}
    E_{K_\alpha} & =  \SI{8.04}{\kilo\electronvolt} \\
    E_{K_\beta}  & =  \SI{8.92}{\kilo\electronvolt} \\
    \symup{\Delta}E_{K_\alpha} &= \SI{203.41}{\electronvolt} \\
    \symup{\Delta}E_{K_\beta}  &= \SI{253.80}{\electronvolt}
\end{align*}
Das Auflösungsvermögen 
\begin{equation}
    A_K = \frac{E_K}{\symup{\Delta}E_K}
\end{equation}
kann daraus zu 
\begin{align*}
    A_{K_\alpha}&= \num{39.55} \\
    A_{K_\beta} &= \num{35.13}
\end{align*}
bestimmt werden.
Mit Hilfe der Formeln $REFERENZ$ lassen sich die Abschirmkonstanten $\sigma_1$, $\sigma_2$ und $\sigma_3$ bestimmen.
Mit $E_\text{abs} = \SI{8987.96(15)}{\electronvolt}$, der Rydberg Energie $R_\infty$ und der Kernladungszahl
$z_\text{Kupfer} = 29 $ lässt sich
\begin{align*}
    \sigma_1 &= \num{3.292(21)}    \\
    \sigma_2 &= \num{12.30(13)}  \\
    \sigma_3 &= \num{22.29(74)}
\end{align*}
errechnen.
\subsection{Bestimmung der Absorptionsenergien und -konstanten von verschiedenen Materialien}
Bei der Aufnahme der Absorptionsspektren wurde ein LiF-Kristall, eine Beschleunigungsspannung von $\SI{35}{\kilo\volt}$ und ein Strom von $\SI{1}{\milli\ampere}$
verwendet. 
In den jeweiligen sind Unterabschnitten sind die Intensitäten $\sfrac{\text{Imp}}{\si{\second}}$ gegen den Winkel $\theta$ aufgetragen.
\subsubsection{Zink}
Bei der Durchführung mit Zink liegt die Mitte der Absorptionskante liegt ungefähr bei $\theta_\text{Zn} \approx \ang{18.75;;}$. Daraus ergibt sich eine Absorptionsenergie bzw. Abschirmkonstante von 
\begin{figure}
    \centering
    \caption{Absorptionsspektrum von Zink}
    \label{fig:zink}
    \includegraphics{build/zink.pdf}
\end{figure}
\begin{align*}
    E_{K, \text{abs}_\text{ Zn}}  &= \SI{9.58}{\kilo\electronvolt} \\
    \sigma_{K_\text{Zn}}         &= \num{3.668} \; \text{.}
\end{align*}
\FloatBarrier
\subsubsection{Gallium}
Die Mitte der Absorptionskante liegt ungefähr bei $\theta_\text{Ga} \approx \ang{17.35;;}$. Die daraus errechnete Energie bzw. Abschirmkonstante beträgt
\begin{figure}
    \centering
    \caption{Absorptionsspektrum von Gallium}
    \label{fig:Gallium}
    \includegraphics{build/gallium.pdf}
\end{figure}
\begin{align*}
    E_{K, \text{abs}_\text{ Ga}}  &= \SI{10.32}{\kilo\electronvolt} \\
    \sigma_{K_\text{Ga}}                &= \num{3.673}
\end{align*}
\FloatBarrier
\subsubsection{Brom}
Bei Brom liegt die Kante ungefähr bei $\theta_\text{Br} \approx \ang{13.25;;}$. Somit lässt sich eine Absorptionsenergie bzw. Abschirmkonstante von 
\begin{figure}
    \centering
    \caption{Absorptionsspektrum von Brom}
    \label{fig:Brom}
    \includegraphics{build/brom.pdf}
\end{figure}
\begin{align*}
    E_{K, \text{abs}_\text{ Br}}  &= \SI{13.43}{\kilo\electronvolt} \\
    \sigma_{K_\text{Br}}                &= \num{3.894}
\end{align*}
errechnen.
\FloatBarrier
\subsubsection{Rubidium}
Die Mitte der Absorptionskante von Rubidium liegt ungefähr bei $\theta_\text{Rb} \approx \ang{11.75;;}$. Daraus ergibt sich eine Absorptionsenergie bzw. Abschirmkonstante von 
\begin{figure}
    \centering
    \caption{Absorptionsspektrum von Rubidium}
    \label{fig:Rubidium}
    \includegraphics{build/rubidium.pdf}
\end{figure}
\begin{align*}
    E_{K, \text{abs}_\text{ Rb}}  &= \SI{15.11}{\kilo\electronvolt} \\
    \sigma_{K_\text{Rb}}         &= \num{4.037} \; \text{.}
\end{align*}
\FloatBarrier
\subsubsection{Strontium}
Die in der Abbildung \ref{fig:Strontium} hervorgehende Mitte der Absorptionskante liegt ungefähr bei $\theta_\text{Sr} \approx \ang{11.05;;}$. 
Daraus ergibt sich eine Absorptionsenergie bzw. Abschirmkonstante von 
\begin{figure}
    \centering
    \caption{Absorptionsspektrum von Strontium}
    \label{fig:Strontium}
    \includegraphics{build/strontium.pdf}
\end{figure}
\begin{align*}
    E_{K, \text{abs}_\text{ Sr}}  &= \SI{16.06}{\kilo\electronvolt} \\
    \sigma_{K_\text{Sr}}         &= \num{4.041} \; \text{.}
\end{align*}
\FloatBarrier
\subsubsection{Zirkonium}
Bei Zirkonium liegt die Mitte der Absorptionskante liegt ungefähr bei $\theta_\text{Zr} \approx \ang{9.95;;}$. 
Daraus ergibt sich eine Absorptionsenergie bzw. Abschirmkonstante von 
\begin{figure}
    \centering
    \caption{Absorptionsspektrum von Zirkonium}
    \label{fig:Zirkonium}
    \includegraphics{build/zirkonium.pdf}
\end{figure}
\begin{align*}
    E_{K, \text{abs}_\text{ Zr}}  &= \SI{17.82}{\kilo\electronvolt} \\
    \sigma_{K_\text{Zr}}         &= \num{4.28} \; \text{.}
\end{align*}
\subsection{Berechnung der Rydbergenergie}
Um die Rydbergfrequenz zu bestimmen kann das Moysley'sche Gesetz mit n = 1 zu 
\begin{equation}
    \sqrt{E_K} = \sqrt{Rh}z - \sqrt{Rh}\sigma_k
\end{equation}
umgestellt werden.
In Regresssionsparameter nimmt die Beziehung die Gestalt
\begin{equation}
    y = mz+b
\end{equation}
an.
Durch Durchführung der linearen Regression lassen sich die Regresssionsparameter zu 
\begin{align*}
        m & = \left( \num{3.568(18)} \right)   \sqrt{\si{\electronvolt}} \\
        b & = \left( \num{-9.086(628)} \right) \sqrt{\si{\electronvolt}}
\end{align*}
\begin{figure}
    \centering
    \caption{Bestimmung der Rydbergfrequenz durch lineare Regression}
    \label{fig:rydberg}
    \includegraphics{build/rydberg.pdf}
\end{figure}
bestimmen.
Daraus kann die Rydbergfrequenz zu 
\begin{equation*}
    R = \SI{3.079(31)}{\hertz}
\end{equation*}
errechnet werden.
Abschließend lässt sich die Rydbergenergie aus der Formel
\begin{equation}
    R_\infty = \symup{h}R
\end{equation}
zu
\begin{equation*}
    R_\infty = \SI{12.74(13)}{\electronvolt}
\end{equation*}
bestimmen.
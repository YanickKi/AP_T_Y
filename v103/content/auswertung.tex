\section{Auswertung}
\label{sec:Auswertung}
Jegliche Fehlerrechnung wurde mit der python-Bibliothek uncertainties \cite{uncertainties} absolviert.
Trotz dessen sind die Formeln für die Unsicherheiten in den jeweiligen Abschnitten angegeben.
Allgemeine Rechnungen wurden mit der python-Bibliothek numpy \cite{numpy} automatisiert. 
Die graphischen Unterstützungen wurden mit Hilfe der python-Bibliothek matplotlib \cite{matplotlib} erstellt.\\
\subsection{Berechnung des Flächenträgheitmoments}
Das Flächenträgheitmoment eines Stabes mit quadratischer Kantenlänge $k$ wird gemäß Beziehung \eqref{eqn:fltm}
\begin{equation}
  I = \int_{-\frac{k}{2}}^{\frac{k}{2}} \int_{-\frac{k}{2}}^{\frac{k}{2}} y^2 \, \symup{d}x\, \symup{d}y
  = \frac{k^3}{12} \int_{-\frac{k}{2}}^{\frac{k}{2}} \, \symup{d}x  = \frac{k^4}{12}
\end{equation}
berechnet.
Das Flächenträgheitmoment eines Stabs mit kreisförmigen Querschnitt mit dem Radius $R$ wird mittels
\begin{equation}
  I = \int_0^{2\pi} \int_0^R r^3  {\sin}^2 \left( \theta \right) \, \symup{d}r \, \symup{d} \theta 
  = \frac{R^4}{8} \int_0^{2\pi} \left(1-2\cos \left( 2\theta \right)\right) \symup{d} \theta = \frac{\pi}{4} R^4
\end{equation}
berechnet.
\subsection{Bestimmung des Elastizitätsmodul bei einseitiger Einspannung} \label{sec:einseitig}
In der Tabelle \ref{tab:einseitig} sind die zu den Weglängen $x$ gemessenen Durchbiegungen $D_1$ bis $D_4$ aufgelistet.
Die Indizes stehen jeweils für die Nummerierung des verwendeten Stabs.
Die zu den Stäben gehörigen Gewichte und Abmessungen sind in der Tabelle \ref{tab:Abmessungen} aufgeführt. 
Dabei steht $h$ für die Höhe, $b$ für die Breite, $L$ für die Länge und $D$ für den Durchmesser, falls der Stab rund ist.
In den Abbildungen \ref{fig:probe1single}, \ref{fig:probe2single}, \ref{fig:probe3single} und \ref{fig:probe4single} sind die 
zu den jeweiligen Stäben gehörenden Messwerte graphisch dargestellt. 
Jedoch sind die Durchbiegungen $D$ nicht gegen $x$ sondern gegen $Lx^2 - \sfrac{x^3}{3}$ aufgetragen, welcher als Linearisierungsterm dient.
Die Gleichung \eqref{eqn:deins} lässt sich in eine Regressionsgerade 
\begin{equation}
  y = m  z + b 
\end{equation}
umschreiben, wobei $y = D$, $m=\sfrac{F}{2EI}$ und $z = Lx^2 - \sfrac{x^3}{3}$ gilt.
Somit lässt sich der Elastizitätsmodul mit $E = \sfrac{F}{2mI}$ ermitteln.
In der Tabelle \ref{tab:regression} sind zu allen Stäben die Regressionsparamter und der damit errechnete Elastizitätsmodul aufgetragen.
\begin{table}
  \centering
  \caption{Messwerte bei einseitiger Einspannung}
  \label{tab:einseitig}
  \begin{tabular}{S[table-format=2.1] S[table-format=1.3] S[table-format=1.3] S[table-format=1.2] S[table-format=1.2]}
  \toprule
  {$x \mathbin{/} \si{\centi\metre}$} & {$D_1 \mathbin{/} \si{\milli\metre}$} & {$D_2 \mathbin{/} \si{\milli\metre}$}
  & {$D_3 \mathbin{/} \si{\milli\metre}$} & {$D_4 \mathbin{/} \si{\milli\metre}$} \\
  \midrule
  5    & 0.07 & 0.075 & 0.17  & 0.15 \\
  10   & 0.06 & 0.2   & 0.52  & 0.58 \\
  15   & 0.03 & 0.39  & 1.7   & 1.11 \\
  20   & 0.03 & 0.49  & 1.76  & 1.95 \\
  25   & 0.1  & 0.99  & 2.58  & 2.62 \\
  30   & 0.14 & 1.33  & 3.56  & 4.05 \\
  35   & 0.255& 1.72  & 4.63  & 4.55 \\
  40   & 1.14 & 2.13  & 5.6   & 6.26 \\
  45   & 1.27 & 2.55  & 6.78  & 6.43 \\
  49.7 & 1.32 & 2.82  & 7.3   & 7.18 \\
  \bottomrule
  \end{tabular}
\end{table}
\begin{table}
  \centering
  \caption{Abmessungen der Stäbe}
  \label{tab:Abmessungen}
  \begin{tabular}{S[table-format=4.0] S[table-format=3.1] S[table-format=1] S[table-format=1] S[table-format=1] S[table-format=2.1]}
  \toprule
  & {$M \mathbin{/} \si{\gram}$} & {$h \mathbin{/} \si{\centi\metre}$} & {$b \mathbin{/} \si{\centi\metre}$}
  & {$D \mathbin{/} \si{\centi\metre}$} & {$L \mathbin{/} \si{\centi\metre}$}\\
  \midrule
  {$\text{Stab 1}$}  & 365.1 & {-} & {-} &  1  & 60.1\\
  {$\text{Stab 2}$}  & 463.7 & 1   & 1   & {-} & 60.1\\
  {$\text{Stab 3}$}  & 166.8 & 1   & 1   & {-} & 60.1\\
  {$\text{Stab 4}$}  & 378.5 & {-} & {-} &  1  & 60.1\\
  \bottomrule
  \end{tabular}
\end{table}
\begin{figure}
  \centering
  \caption{Messwerte und Regressionsgerade der Probe 1}
  \label{fig:probe1single}
  \includegraphics{build/probe1single.pdf}
\end{figure}
\begin{figure}
  \centering
  \caption{Messwerte und Regressionsgerade der Probe 2}
  \label{fig:probe2single}
  \includegraphics{build/probe2single.pdf}
\end{figure}
\begin{figure}
  \centering
  \caption{Messwerte und Regressionsgerade der Probe 3}
  \label{fig:probe3single}
  \includegraphics{build/probe3single.pdf}
\end{figure}
\begin{figure}
  \centering
  \caption{Messwerte und Regressionsgerade der Probe 4}
  \label{fig:probe4single}
  \includegraphics{build/probe4single.pdf}
\end{figure}
\begin{table}
  \centering
  \caption{Regressionsparamter und Elastizitätsmodul der Stäbe}
  \label{tab:regression}
  \begin{tabular} {S[table-format=3.0] 
    S[table-format=1.2] @{${}\pm{}$} S[table-format=1.2]
    S[table-format=1.2] @{${}\pm{}$} S[table-format=1.2] 
    S[table-format=3.2] @{${}\pm{}$} S[table-format=2.2]}
  \toprule
  & \multicolumn{2}{c}{$m \mathbin{/} \si{\metre\tothe{-2}} \cdot \num{e-3}$} & 
    \multicolumn{2}{c}{$b \mathbin{/} \si{\metre} \cdot \num{e-4}$} & 
    \multicolumn{2}{c}{$E \mathbin{/} \si{\giga\pascal}$}\\
  \midrule
  {$\text{Stab 1}$}  & 1.37 & 0.22 & 1.79 & 1.25 & 759.77 & 12.08\\
  {$\text{Stab 2}$}  & 2.69 & 0.07 & 0.54 & 0.42 & 228.52 & 6.29  \\
  {$\text{Stab 3}$}  & 6.8  & 0.26 & 3.88 & 1.47 & 90.4   & 3.41 \\ 
  {$\text{Stab 4}$}  & 6.87 & 0.38 & 3.89 & 2.19 & 152.09 & 8.47 \\
  \bottomrule
  \end{tabular}
\end{table}
Der Fehler des Elastizitätsmoduls ergibt sich nach Gauß durch
\begin{equation}
  \symup{\Delta} E = \left | \frac{\partial E}{\partial m}  \symup{\Delta} m \right | \;\text{.} \label{eqn:gauss}
\end{equation}
\FloatBarrier
\subsection{Bestimmung des Elastizitätsmodul bei beidseitiger Einspannung}
In der Tabelle \ref{tab:beidseitigrechts} sind die gemessenen Durchbiegungen $D$ für gemessene Weglängen $x_R$, welche sich auf 
die rechte Seite beziehen, aufgeführt.
In der Tabelle \ref{tab:beidseitiglinks} sind die Messwerte analog zur ebend genannten Tabelle für die linke Seite aufgelistet.
Bei den Messwerten rechts von der mitte ($0 \leq x \leq \sfrac{L}{2}$) lautet der Linearisierungsterm $3L^2x-4x^3$.
Die linken Messwerte ($\sfrac{L}{2} \leq x \leq L$) werden mit Hilfe von $4x^3-12Lx^2+9L^2x-L^3$ linearisiert.
Zu den jeweiligen Stäben wurden die Abbildungen \ref{fig:probe1double}, \ref{fig:probe2double}, \ref{fig:probe3double} und \ref{fig:probe4double} erstellt,
welche die Messwerte und die Regressionsgeraden enthalten.
Die Regressionsvorschrift für die rechten und linken Messwerte lautet ähnlich wie in Abschnitt \ref{sec:einseitig}
\begin{equation}
  y = m z +b \; \text{.}
\end{equation}
wobei $m = \sfrac{F}{48EI}$ gilt. Je nach Linearisierung gilt $z = 3L^2x-4x^3$ oder $z = 4x^3-12Lx^2+9L^2x-L^3$.
Nach Durchführung der Regression lässt sich der Elastizitätsmodul mit Hilfe von $E = \sfrac{F}{48mI}$ ermitteln.
In der Tabelle \ref{tab:regressiondouble} sind die Regressionsparameter und der dazugehörige Elastizitätsmodul für die rechte Seite aufgelistet.
Für die linke Seite dient die Tabelle \ref{tab:regressiondoublelinks}.
Die Unsicherheit der beiden Elastizitätsmodulen wird genau so wie in Abschnitt \ref{sec:einseitig} mit Hilfe von der 
Gleichung \eqref{eqn:gauss} berechnet.
\begin{table}
  \centering
  \caption{Messwerte bei beidseitiger Einspannung für {$0 \leq x \leq \sfrac{L}{2}$}}
  \label{tab:beidseitigrechts}
  \begin{tabular}{S[table-format=2.1] S[table-format=1.3] S[table-format=1.3] S[table-format=1.2] S[table-format=1.2]}
  \toprule
  {$x_\text{R} \mathbin{/} \si{\centi\metre}$} & {$D_1 \mathbin{/} \si{\milli\metre}$} & {$D_2 \mathbin{/} \si{\milli\metre}$}
  & {$D_3 \mathbin{/} \si{\milli\metre}$} & {$D_4 \mathbin{/} \si{\milli\metre}$} \\
  \midrule
  5    & 0.03 & 0.04 & 0.38  & 0.06 \\
  10   & 0.09 & 0.12 & 0.65  & 0.22 \\
  15   & 0.17 & 0.17 & 0.93  & 0.41 \\
  20   & 0.26 & 0.2  & 1.12  & 0.59 \\
  25   & 0.35 & 0.26 & 1.15  & 0.75 \\
  \bottomrule
  \end{tabular}
\end{table}
\begin{table}
  \centering
  \caption{Messwerte bei beidseitiger Einspannung für {$\sfrac{L}{2} \leq x \leq L$}}
  \label{tab:beidseitiglinks}
  \begin{tabular}{S[table-format=2.1] S[table-format=1.3] S[table-format=1.3] S[table-format=1.2] S[table-format=1.2]}
  \toprule
  {$x_\text{L} \mathbin{/} \si{\centi\metre}$} & {$D_1 \mathbin{/} \si{\milli\metre}$} & {$D_2 \mathbin{/} \si{\milli\metre}$}
  & {$D_3 \mathbin{/} \si{\milli\metre}$} & {$D_4 \mathbin{/} \si{\milli\metre}$} \\
  \midrule
  55   & 0.02 & 0.07 & 0.03  & 0.12 \\
  50   & 0.14 & 0.14 & 0.39  & 0.25 \\
  45   & 0.26 & 0.22 & 0.71  & 0.46 \\
  40   & 0.35 & 0.25 & 0.98  & 0.66 \\
  35   & 0.36 & 0.27 & 1.1   & 0.81 \\
  \bottomrule
  \end{tabular}
\end{table}
\begin{figure}
  \centering
  \includegraphics{build/probe1double.pdf}
  \caption{Messwerte und Regressionsgerade der Probe 1}
  \label{fig:probe1double}
\end{figure}
\begin{figure}
  \centering
  \includegraphics{build/probe2double.pdf}
  \caption{Messwerte und Regressionsgerade der Probe 2}
  \label{fig:probe2double}
\end{figure}
\begin{figure}
  \centering
  \includegraphics{build/probe3double.pdf}
  \caption{Messwerte und Regressionsgerade der Probe 3}
  \label{fig:probe3double}
\end{figure}
\begin{figure}
  \centering
  \includegraphics{build/probe4double.pdf}
  \caption{Messwerte und Regressionsgerade der Probe 4}
  \label{fig:probe4double}
\end{figure}
\begin{table}
  \centering
  \caption{Regressionsparameter und Elastizitätsmodul der Stäbe (rechts)}
  \label{tab:regressiondouble}
  \begin{tabular} {S[table-format=3.0] 
    S[table-format=1.2] @{${}\pm{}$} S[table-format=1.2]
    S[table-format=1.2] @{${}\pm{}$} S[table-format=1.2]  
    S[table-format=3.2] @{${}\pm{}$} S[table-format=2.2]}
  \toprule
  & \multicolumn{2}{c}{$m_\text{R} \mathbin{/} \si{\metre\tothe{-2}} \cdot \num{e-3}$} & 
    \multicolumn{2}{c}{$b_\text{R} \mathbin{/} \si{\metre} \cdot \num{e-4}$} & 
    \multicolumn{2}{c}{$E_\text{R} \mathbin{/} \si{\giga\pascal}$}\\
  \midrule
  {$\text{Stab 1}$}  & 2.02 & 0.24 & -1.02 & 0.37 & 485.49 & 59.40 \\
  {$\text{Stab 2}$}  & 1.32 & 0.11 & -0.27 & 0.16 & 436.15 & 35.01 \\
  {$\text{Stab 3}$}  & 5.23 & 0.36 & 1.13  & 0.54 & 110.04 & 7.67  \\ 
  {$\text{Stab 4}$}  & 4.41 & 0.34 & 2.11  & 0.51 & 222.31 & 2.05  \\
  \bottomrule
  \end{tabular}
\end{table}
\begin{table}
  \centering
  \caption{Regressionsparameter und Elastizitätsmodul der Stäbe (links)}
  \label{tab:regressiondoublelinks}
  \begin{tabular} {S[table-format=3.0]  
    S[table-format=1.2] @{${}\pm{}$} S[table-format=1.2]
    S[table-format=1.2] @{${}\pm{}$} S[table-format=1.2]
    S[table-format=3.2] @{${}\pm{}$} S[table-format=2.2]}
  \toprule
  & \multicolumn{2}{c}{$m_\text{L} \mathbin{/} \si{\metre\tothe{-2}} \cdot \num{e-3}$} & 
    \multicolumn{2}{c}{$b_\text{L} \mathbin{/} \si{\metre} \cdot \num{e-4}$} &
    \multicolumn{2}{c}{$E_\text{L} \mathbin{/} \si{\giga\pascal}$}\\
  \midrule
  {$\text{Stab 1}$} & 2.32 & 0.16 & -1.01 & 0.25 & 421.36 & 29.32 \\
  {$\text{Stab 2}$} & 1.33 & 0.1  & 0.02  & 0.15 & 432.82 & 32.72\\
  {$\text{Stab 3}$} & 7.06 & 0.15 & -3.52 & 0.23 & 81.78  & 7.67  \\ 
  {$\text{Stab 4}$} & 4.52 & 0.42 & -1.77 & 0.65 & 216.84 & 2.05  \\
  \bottomrule
  \end{tabular}
\end{table}
\section{Durchführung}
\label{sec:Durchführung}
Um die Zeitabhängigkeit der Amplitude zu untersuchen wurde der Versuch wie in Abbildung \ref{fig:aufb a} aufgebaut.
\begin{figure}
    \centering
    \caption{Aufbau zur Untersuchung einer ungedämften und erzwungenen Schwingung \cite{v354}} 
    \label{fig:aufb a}
    \includegraphics[width = 0.5\textwidth]{pics/aufb.png}
\end{figure}
Dabei wurde an Stelle des Nadelimpulsgenerators ein Rechtecksignal genutzt. 
Der RLC-Kreis wird mit niederfrequenten Rechteckimpulsen ausgelegt und zum Schwingen angeregt.
Es werden die Amplituden der Spannung $U_\text{C}$ in Abhängigkeit von der Zeit $t$ der gedämpften Schwingung ausgelesen und notiert.
\\
Die Durchführung des zweiten Versuchsteils erfolgt mit dem selben Aufbau wie in Abbildung \ref{fig:aufb a}.
Jedoch wird der Widerstand durch einen verstellbaren ersetzt. Danach wird versucht durch Variation des Widerstands den aperiodischen Grenzfall zu erreichen.
Sobald dieser erreicht ist, wird der Wert für den Widerstand notiert.
\\
Für den dritten Versuchsteil wurde der Versuch erneut wie in Abbildung \ref{fig:aufb a} aufgebaut. Dies mal wird nur die Rechteckspannung durch eine Sinusspannung ersetzt.
Die Frequenzabhängigkeit der Kondensatorspannung $U_\text{C}$ wird untersucht, indem die Frequenz von $\SI{5}{\kilo \hertz}$ bis $\SI{60}{\kilo \hertz}$ erhöht wird.


\section{Diskussion}
\label{sec:Diskussion}
Bei Betrachtung der Güte fällt es auf, dass diese stark von der theoretischen  Güte abweicht.
Die Abweichung beträgt
\begin{equation}
    \frac{ q_\text{Theo} - q}{q_\text{Theo}} = \SI{38.69}{\percent} \; \text{,}
\end{equation}
welche eine relativ hohe Abweichung ist. 
Möglicherweiser könnte es daran liegen, dass bei der Berechnung der theoretischen Güte der Innenwiderstand des
Sinusgenerators vernachlässigt wurde. 
Da durch den Widerstand geteilt wird, liegt dies nahe.
Wo sich der vernachlässigte Widerstand ebenfalls wiederspiegelt ist bei der Berechung der Frequenz, wo die Spannung auf ein Faktor 
$\sfrac{U_\text{max}}{\sqrt{2}}$ abgefallen ist. Denn die daraus errechente theoretische Breite ähnelt der experimentell bestimmen Breite ähnlich wenig.
Außerdem sind die Frequenzen $\nu_+$ und $\nu_-$ im Vergleich zu den theoretisch berechneten Frequenzen um ca. $\SI{1}{\kilo\hertz}$ verschoben.
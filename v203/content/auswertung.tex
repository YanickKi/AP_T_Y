\section{Auswertung}
\label{sec:Auswertung}
In der Tabelle \ref{tab:tiefdruck} sind die gemessenen Temperaturen $T_\text{g}$ und $T_\text{f}$ mit dem dazugehörigen Druck $p$
aufgetragen. Dabei stehen die Indizes für den gasförmigen bzw. flüssigen Zustand.
\subsection{Berechnung der gemittelten Verdampfungswärme von Wasser}
\begin{table}
    \centering
    \caption{Gemessener Druck $p$ bei den Temperaturen $T_\text{f}$ und $T_\text{g}$.}
    \label{tab:tiefdruck}
    \begin{tabular} {S[table-format=2.0] S[table-format=3.1] S[table-format=3.0]}
        \toprule
        {$T_f \mathbin{/} \si{\celsius}$} & {$T_g \mathbin{/} \si{\celsius}$} & {$p \mathbin{/} \si{\milli\bar}$}\\
    \midrule
    30   &   25   &74  \\
    35   &   36   &86  \\
    40   &   38   &96  \\
    45   &   40   &110 \\
    50   &   49   &127 \\
    55   &   54   &160 \\
    60   &   59.5 &200 \\
    65   &   64   &247 \\
    70   &   70   &307 \\
    75   &   71   &372 \\
    80   &   82   &460 \\
    85   &   99   &550 \\
    90   &   106.5&648 \\
    \bottomrule
\end{tabular}
\end{table}
Dazu ist der durch den Normaldruck $p_0 \approx \SI{1}{\bar}$ geteilte Druck $p$ logarithmisch gegen die reziproke Absoluttemperatur 
$T_\text{g}$ in Der Abbildung \ref{fig:tiefdruck} graphisch dargestellt. 
\begin{figure}
    \centering
    \caption{Der gemessene Druck gegen die reziproke absolute Gastemperatur.}
    \label{fig:tiefdruck}
    \includegraphics{build/tiefdruck.pdf}
\end{figure}
Die Beziehung $REFERENZ$ lässt sich in eine Regressionsgerade der Form 
\begin{equation*}
    y = mx+b
\end{equation*}
umschreiben, wobei $y=\ln \left ( \sfrac{p}{p_0} \right)$ und $m=\sfrac{-L}{R}$ gilt.
Dabei ergeben sich die Parameter zu 
\begin{align*}
    m &= \SI{-3369.90(17463)}{\mole\kelvin}\\
    b &= \SI{0.65(53)} \; \text{.}
\end{align*}
Die universelle Gaskonstante hat den Wert $R = \SI{8.31446261815324}{\kilo\gram\metre\squared\per\second\squared\per\mole\per\kelvin}$\cite{gasconstant}, womit sich
eine Verdampfungswärme von 
\begin{equation*}
    L = \SI{2.80(15)e04}{\joule\per\mole}
\end{equation*}
errechnen lässt.
\subsection{Bestimmung der Zeitabhängigkeit der Verdampfungswärme}
In der Tabelle \ref{tab:hochdruck} sind der gemessene Druck und die Temperatur für den Bereich $p > \SI{1}{\bar}$ aufeführt.
\begin{table}
    \centering
    \caption{Gemessener Druck $p$ bei den Temperaturen $T_\text{f}$ und $T_\text{g}$.}
    \label{tab:hochdruck}
    \begin{tabular} {S[table-format=2.0] S[table-format=3.1] S[table-format=3.0]}
        \toprule
        {$T_f \mathbin{/} \si{\celsius}$} & {$T_g \mathbin{/} \si{\celsius}$} & {$p \mathbin{/} \si{\milli\bar}$}\\
    \midrule
    30   &   25   &74  \\
    35   &   36   &86  \\
    40   &   38   &96  \\
    45   &   40   &110 \\
    50   &   49   &127 \\
    55   &   54   &160 \\
    60   &   59.5 &200 \\
    65   &   64   &247 \\
    70   &   70   &307 \\
    75   &   71   &372 \\
    80   &   82   &460 \\
    85   &   99   &550 \\
    90   &   106.5&648 \\
    \bottomrule
\end{tabular}
\end{table}
Die Messwerte sind dazu in der Abbildung \ref{fig:hochdruck} graphisch dargestellt.
Die Regressionskurve wird durch ein Polynom dritten Grads der Form 
\begin{equation*}
    y = ax^3+bx^2+cx+d    
\end{equation*}
errechnet, wobei sich die Parameter zu
\begin{align*}
    a &= \SI{1.93(50)}{\pascal\per\kelvin\tothe{3}}                 \\
    b &= \SI{-2299.79(64999)}{\pascal\per\kelvin\tothe{2}}          \\
    c &= \SI{923918.92(28013445)}{\pascal\per\kelvin}     \\
    d &= \SI{-124836792.62(4018067192)}{\pascal\per\kelvin} \; \text{.}
\end{align*}
Diese Regressionskurve wird nach T abgeleitet und in die nach L umgestellte Clausius-Clapeyronsche Gleichung $REFERENZ$ eingesetzt.
Jedoch kann $V_\text{D}$ nicht mehr mittels der allgemeinen Gasgleichung bestimmt werden sondern muss mittels der Näherung
\begin{equation*}
    \left( p + \frac{a}{V^2}\right) V = RT
\end{equation*}
bestimmt werden.
Das Volumen des in dem flüssigen Zustand kann erneut vernachlässigt werden.
Somit ergibt sich für die Verdampfungswärme die Zeitabhängigkeit
\begin{equation*}
    L = \left( \frac{RT}{2p} \pm \sqrt{\frac{R^2T^2}{4p^2} + \frac{a}{p}}- V_\text{F}\right)\left( 5.79 T^3 + 4599.58 T^2 + 923918.92T \right) \; \text{.}
\end{equation*} 
wobei $a = \SI{0.9}{\joule\metre\tothe{3}\per\mole\squared}$ gilt.
\begin{figure}
    \centering
    \caption{Gegen die Temperatur aufgetragener Druck.}
    \label{fig:hochdruck}
    \includegraphics{build/hochdruck.pdf}
\end{figure}

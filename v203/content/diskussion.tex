\section{Diskussion}
\label{sec:Diskussion}
In der Abbildung \ref{fig:tiefdruck} wird ersichtlich, dass die Messwerte eine relativ hohe Abweichung aufweisen.
Diese Beobachtung wird durch die hohen Unsicherheiten der Regressionsparameter gestärkt.
Gründe hierfür könnte die gleichzeitige Messung von drei Messwerten sein, welche der Druck, die Temperatur des flüssigen Zustands und die Temperatur des Gases 
waren.
Da diese nicht gleichzeitig abgelesen und notiert wurden, sondern nur nacheinander abgelesen wurden, konnte eine der Größen, während eine andere Größe gemessen wurde,
weiter steigen.
Ein weiterer Faktor könnten die Theromstate sein, da die Ableseganaugkeit relativ gering ist, da dort per Augenmaß erfasst werden muss, welche Temperatur angezeigt wird.
Ebenfalls bieten die Theromstate kein allzu hohes Auflösungsvermögen, was weniger Präzision bietet.
Die errechnete Verdampfungswärme weicht von dem Literaturwert\cite{lit} bei $\SI{100}{\celsius}$ um
\begin{equation*}
    \frac{L_\text{i, lit} - L_\text{i}}{L_\text{i, lit}} = \frac{\SI{4.07e04}{\joule\per\mole} - \SI{2.49e04}{\joule\per\mole}}{\SI{4.07e04}{\joule\per\mole}} = \SI{38.82}{\percent}
\end{equation*}
ab.
Die experimentell errechnete Verdampfungswärme wurde über mehrere Temperaturen gemittelt, wohingegen der Literaturwert sich nur auf $\SI{100}{\celsius}$ bezieht,
wodurch die relativ hohen Abweichungen zu Stande gekommen sein könnten.
Zu der Bestimmung der Temperaturabhängigkeit der Verdampfungswärme kann gesagt werden, dass dort mit einer hohen Präzision gemessen wurde, da die Regressionsparameter
kleine Unsicherheiten aufweisen.
Ebenfalls liegen die Messwerte sehr nah an der Regressionskurve, was die hohe Genauigkeit bestätigt.
\section{Diskussion}
\label{sec:Diskussion}
In Abschnitt \ref{sec:phase} wird bereits bei den Instantanaufnahmen ersichtlich, dass die Spannungsverläufe mit dem verrauschten Signal
verrauscht erscheinen. 
Ebenfalls wird auffällig, dass die Spannungsverläufe um die x-Achse gedreht werden und die Amplitude des verrauschten Signals merkbar kleiner ist.
Bei Betrachtung der Abbildungen \ref{fig:Uwo} und \ref{fig:Uwi} fällt auf, dass die Regressionskurven sehr ähnlich verlaufen, während die Messwerte 
des unverrauschten Signals eine etwas höhere Abweichung aufweisen, was sich in den Unsicherheiten der Regressionsparameter
wiederspiegelt.
Die Beziehung $U \propto U_0 \cos (\phi )$ konnte in diesem Versuchsteil gut überprüft werden.\\
In der Abbildung \ref{fig:distance} erscheint es so, als ob die gemessene Intensität bzw. Spannung eine 
$\sfrac{1}{r}$ -Abhängigkeit aufweist, da die Regressionskurve mit dieser Abhängigkeit die Messwerte besser darstellt.
Jedoch kann der Abbildung der Fakt, dass die Intensität niemals 0 sein wird, entnommen werden.
Dies liegt daran, dass der Raum nicht komplett abgedunkelt wurde und somit immer eine Grundintensität vorhaden ist.
Jedoch kann dort erkannt werden, dass die Intensität der LED ab einem Abstand von ca. $r_\text{max} = \SI{40}{\centi\metre}$ nicht mehr gemessen wurde. 
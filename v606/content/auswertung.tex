\section{Auswertung}
\label{sec:Auswertung}
Jegliche Fehlerrechnung wurde mit der python-Bibliothek uncertainties \cite{uncertainties} absolviert.
Allgemeine Rechnungen wurden mit der python-Bibliothek numpy \cite{numpy} automatisiert. 
Die graphischen Unterstützungen wurden mit Hilfe der python-Bibliothek matplotlib \cite{matplotlib} erstellt.\\
\subsection{Untersuchung der Filterkurve des Selektiv-Verstärkers}
\label{sec:filter}
Die Messwerte zur Untersuchung der Filterkurve sind in der Tabelle \ref{tab:frequence} aufgetragen. 
Diese enthält die gemessene Frequenz und  Ausgangsspannung $U_\text{A}$ und den daraus berechneten Quotienten $\sfrac{U_\text{A}}{U_\text{E}}$, welcher sich aus der
eben genannten Ausgangsspannung und der konstanten Eingangsspannung $U_\text{E} = \SI{0.01}{\volt}$ zusammensetzt.
\begin{table}
    \centering
    \caption{Gemessene Ausgangsspannung und der daraus berechnete Quotient $U = \sfrac{U_\text{A}}{U_\text{E}}$ in Abhängigkeit von der Frequenz}
    \label{tab:frequence}
    \begin{tabular} {S[table-format=2.2] S[table-format=1.4] S[table-format=1.2] S[table-format=2.2] S[table-format=1.4] S[table-format=1.2]}
        \toprule
        {$\nu \mathbin{/} \si{\kilo\hertz}$} & {$U_\text{A} \mathbin{/} \si{\volt}$} & {$U$} &
        {$\nu \mathbin{/} \si{\kilo\hertz}$} & {$U_\text{A} \mathbin{/} \si{\volt}$} & {$U$} \\
    \midrule
    20.00 & 0.0085 & 0.85 & 34.60 & 0.0670 & 6.70\\
    21.00 & 0.0095 & 0.95 & 34.61 & 0.0690 & 6.90\\
    22.00 & 0.0003 & 0.03 & 34.62 & 0.0710 & 7.10\\
    23.00 & 0.0004 & 0.04 & 34.63 & 0.0720 & 7.20\\
    24.00 & 0.0006 & 0.06 & 34.64 & 0.0740 & 7.40\\
    25.00 & 0.0007 & 0.07 & 34.65 & 0.0760 & 7.60\\
    26.00 & 0.0010 & 0.10 & 34.66 & 0.0780 & 7.80\\
    27.00 & 0.0030 & 0.30 & 34.67 & 0.0800 & 8.00\\
    28.00 & 0.0160 & 1.60 & 34.70 & 0.0840 & 8.40\\
    29.00 & 0.0210 & 2.10 & 34.75 & 0.0870 & 8.70\\
    30.00 & 0.0280 & 2.80 & 34.77 & 0.0870 & 8.70\\
    31.00 & 0.0390 & 3.90 & 34.80 & 0.0850 & 8.50\\
    32.00 & 0.0565 & 5.65 & 34.81 & 0.0825 & 8.25\\
    33.00 & 0.0940 & 9.40 & 34.82 & 0.0820 & 8.20\\
    34.00 & 0.0220 & 2.20 & 34.83 & 0.0810 & 8.10\\
    34.10 & 0.0190 & 1.90 & 34.84 & 0.0800 & 8.00\\
    34.20 & 0.0230 & 2.30 & 34.85 & 0.0770 & 7.70\\
    34.30 & 0.0290 & 2.90 & 34.86 & 0.0750 & 7.50\\
    34.40 & 0.0380 & 3.80 & 34.87 & 0.0740 & 7.40\\
    34.50 & 0.0500 & 5.00 & 34.88 & 0.0720 & 7.20\\
    34.51 & 0.0510 & 5.10 & 34.89 & 0.0700 & 7.00\\
    34.52 & 0.0530 & 5.30 & 34.90 & 0.0680 & 6.80\\
    34.53 & 0.0540 & 5.40 & 35.00 & 0.0510 & 5.10\\
    34.54 & 0.0560 & 5.60 & 36.00 & 0.0650 & 6.50\\
    34.55 & 0.0580 & 5.80 & 37.00 & 0.0070 & 0.70\\
    34.56 & 0.0590 & 5.90 & 38.00 & 0.0052 & 0.52\\
    34.57 & 0.0610 & 6.10 & 39.00 & 0.0038 & 0.38\\
    34.58 & 0.0630 & 6.30 & 40.00 & 0.0030 & 0.30\\
    34.59 & 0.0650 & 6.50 & { }   & { }    & { } \\
    \bottomrule
\end{tabular}
\end{table}
Mit Hilfe der Messwerte lassen sich die Abbildungen \ref{fig:frequence} und \ref{fig:frequencefine} erstellen, welche den gegen die Frequenz
aufgetragenen Spannungsquotienten darstellen.
In der Abbildung \ref{fig:frequence} ist das gesamte gemessene Frequenzspektrum von $\SI{20}{\kilo\hertz}$ bis $\SI{40}{\kilo\hertz}$ aufgetragen.
Jedoch lässt sich dort nicht viel über das Verhalten der Spannung in der Nähe der Resonanzfrequenz erkennen, da nicht nur ein paar Messwerte abweichen, sondern eine
systematische Abweichung vorliegt. Die möglichen Ursachen werden in der Diskussion(\ref{sec:Diskussion}) näher erläutert.  
Dazu wurde die Abbildung \ref{fig:frequencefine} erstellt, welche den Bereich um  die Resonanzfrequenz besser darstellt und auch eine bessere 
Rekursionskurve ermöglicht.
Die für die Regresson nötige Abbildungsvorschrift ist 
\begin{equation}
    U (\nu) = \frac{c}{\left( \nu^2 - \nu_0^2 \right)^2 + d^2 \nu_0^2} \; \text{.}
\end{equation}
Mittels Regressionsrechnung ergeben sich die Regressionsparameter zu 
\begin{align*}
    \nu_0 & =\num{34.755(2)}       \\
    c     & =\num{3741.23(96067)}  \\
    d     & =\num{0.602(9)} \; \text{.}
\end{align*}
Um nun die Güte berechnen zu können, müssen die Frequenzen $\nu_-$ und $\nu_+$, bei denen die Spannung $U$ den Wert $\sfrac{U_\text{max}}{\sqrt{2}}$ annimmt, bestimmt werden.
Mittels python lassen sich diese zu
\begin{align*}
    \nu_- &= \SI{34.56}{\kilo\hertz} \\
    \nu_+ &= \SI{34.949}{\kilo\hertz}
\end{align*}
bestimmen, woraus mit Hilfe der Beziehung $Q = \sfrac{\nu_0}{\nu_+ - \nu_-}$ eine experimentell bestimmte Güte von 
\begin{equation*}
    Q_\text{Exp} = \num{89.508(5)}
\end{equation*}
errechnet werden kann.
\begin{figure}
    \centering
    \caption{Verhältnis der Ausgangs- und Eingangsspannung}
    \label{fig:frequence}
    \includegraphics{build/frequence.pdf}
\end{figure}
\begin{figure}
    \centering
    \caption{2.Verhältnis der Ausgangs- und Eingangsspannung im Bereich um die Resonanzfrequenz}
    \label{fig:frequencefine}
    \includegraphics{build/frequencefine.pdf}
\end{figure}
\FloatBarrier
\subsection{Theoretische Bestimmung der Suszeptibilität von $\ce{Dy2O3}$ und $\ce{Nd2O3}$}
Um die Suszeptibilitäten theoretisch bestimmen zu können, müssen zunächst der Spin, der Bahndrehimpuls und der Gesamtdrehimpuls mit Hilfe der Hundschen 
Regeln bestimmt werden.
Mit Hilfe der magnetischen Quantenzahl $m_l$ lässt sich der Gesamtbahndrehimpuls bestimmen, welche von $-l$ bis $l$ geht, wobei l die 
Nebenquantenzahl ist, welche bei der 4f-Schale den Betrag $3$ besitzt. 
Damit der Gesamtbahndrehimpuls maximal ist, werden die magnetischen Quantenzahlen $m_l$ addiert. 
Um alle Elektronen zu beschreiben, müssen die größten magnetischen Quantenzahlen wieder abgezogen werden.
Zur Berechnung des Gesamtbahndrehimpulses der 4f-Elektronen von $\ce{Dy2O3}$ wird 
\begin{equation*}
    \lvert L \rvert = \lvert -3 - 2 - 1 + 0 + 1 + 2 + 3 - 3 - 2 \rvert = 5
\end{equation*} 
gerechnet. 
Analog dazu wird der Spin mittels
\begin{equation*}
    S = 7 \cdot \frac{1}{2} - 2  \cdot \frac{1}{2}
\end{equation*}
berechnet. Daraus kann der Gesamtdrehimpuls mittels $J = L \pm s$ errechnet werden.
Nach der Berechnung von $g_J$ lässt sich die Suszeptibilität mit Hilfe der Gleichung \eqref{eqn:sustheo} errechnen.
Die Momentanzahl pro Volumeneinheit $N$ wird durch
\begin{equation}
    N = 2\cdot\frac{\rho}{M}\cdot N_A 
\end{equation}
bestimmt,
wobei $N_A$ die Avogadro-Konstante mit $N_A = \SI{6.02214076e23}{\per\mole}$\cite{avogrado}, $\rho$ die Dichte $M$ die molare Masse 
\footnote{Die molare Masse von $\ce{Dy2O3}$ und $\ce{Nd2O3}$ wurden den Quellen \cite{dy} und \cite{nd} entnommen.} der Probe ist. 
Das Borsche Magneton hat den Wert $\mu_B = \SI{9.2740100783e-24}{\joule\per\tesla}$\cite{magneton} und die magnetische Feldkonstante den Wert
$\mu_0 = \SI{1.25663706212e-6}{\newton\per\ampere\squared}$\cite{magnet}.
Die Boltzmann-Konstante beträgt $k_\text{B} = \SI{1.380649e-23}{\joule\per\kelvin}$\cite{boltzmann}.
Als Temperatur wird eine Raumtemperatur von $T = \SI{293.15}{\kelvin}$ angenommen.
Daraus lässt sich die Tabelle \ref{tab:susztheo} erstellen, welche die für die Berechnung der Suszeptibilität nötigen Daten hat.
\begin{table}
    \centering
    \caption{Probenspezfische Daten und die theoretische Suszeptibilität}
    \label{tab:susztheo}
    \begin{tabular} {S[table-format=4.0] S[table-format=1.2]
                     S[table-format=3.3] S[table-format=1.2]
                     S[table-format=1.0] S[table-format=1.0]
                     S[table-format=1.1] S[table-format=1.2]}
        \toprule
        {$\text{Probe}$} & {$\rho \mathbin{/} \si{\gram\centi\metre\tothe{-1}}$} &
        {$M \mathbin{/} \si{\gram\mole\tothe{-3}}$} & {$N \mathbin{/} \SI{e28}{\metre\tothe{-1}}$} &
        {$S$} & {$L$} & 
        {$J$} & {$\chi \cdot \num{e-2}$} \\
    \midrule
    {$\ce{Dy2O3}$} & 7.8  & 372.998 & 2.52 & {$\sfrac{5}{2}$} & 5 & 7.5 & 2.54\\
    {$\ce{Nd2O3}$} & 7.24 & 336.48  & 2.59 & {$\sfrac{3}{2}$} & 6 & 4.5 & 0.25\\
    \bottomrule 
\end{tabular}
\end{table}
\subsection{Experimentelle Bestimmung der Suszeptibilität von $\ce{Dy2O3}$ und $\ce{Nd2O3}$}
Um die Suszeptibilität von den beiden Proben bestimmen zu können, muss der reale Querschnitt berechnet werden, da sich die Proben nicht beliebig stopfen lassen.
Dieser wird gemäß der Gleichung
\begin{equation}
    Q_\text{real} = \frac{M}{L\rho}
\end{equation}
berechnet, wobei $M$ die masse, $L$ die Länge der Probe, welche in der Spule ist und $\rho$ die Dichte des Materials.
In der Tabelle sind die Eigenschaften und der reale Querschnitt für die beiden Proben aufgezeigt.
\begin{table}
    \centering
    \caption{Relevante Daten zur Bestimmung des realen Querschnitts von $\ce{Dy2O3}$ und $\ce{Nd2O3}$}
    \label{tab:info}
    \begin{tabular} {S[table-format=3.0] S[table-format=2.2] S[table-format=1.2] S[table-format=2.2] S[table-format=2.2]}
        \toprule
    {$\text{Probe}$} & {$M \mathbin{/} \si{\gram}$} & {$\rho \mathbin{/} \si{\gram\centi\metre\tothe{-3}}$} & {$L \mathbin{/} \si{\centi\metre}$} &
    {$Q_\text{real} \mathbin{/} \si{\milli\metre}$}\\
    \midrule
    {$\ce{Dy2O3}$} & 14.38 & 7.8  & 15.5 & 11.89\\
    {$\ce{Nd2O3}$} & 9.2   & 7.24 & 15.7 & 8.09 \\
    \bottomrule
\end{tabular}
\end{table}
In der Tabelle \ref{tab:messdy} und \ref{tab:messnd} sind die gemessenen Spannungen und Widerstände nach dem Abgleichen ohne bzw. mit eingeführter Probe
und die daraus resultierende Widerstandsdifferenz aufgeführt.
\begin{table}
    \centering
    \caption{Gemessene Widerstände und Spannung und die daraus berechnete Widerstands- und Spannungsdifferenz von \ce{Dy2O3}}
    \label{tab:messdy}
    \begin{tabular} {S[table-format=1.1] S[table-format=4.0] S[table-format=1.1] S[table-format=3.0] S[table-format=1.3] S[table-format=1.1]}
        \toprule
        {$U_\text{ohne} \mathbin{/} \si{\milli\volt}$} & {$R_\text{ohne} \mathbin{/} \si{\milli\ohm}$} & {$U_\text{mit} \mathbin{/} \si{\milli\volt}$}    &
        {$R_\text{mit} \mathbin{/} \si{\milli\ohm}$}   & {$\symup{\Delta}R \mathbin{/} \si{\ohm}$} & {$\symup{\Delta}U \mathbin{/} \si{\milli\volt}$}\\
    \midrule
    3.5 & 2200 & 7.5 & 565 & 1.635 & 4.0 \\
    3.4 & 2200 & 7.5 & 600 & 1.600 & 4.1 \\
    3.5 & 2200 & 8.0 & 450 & 1.750 & 4.5 \\
    \bottomrule
\end{tabular}
\end{table}
\begin{table}
    \centering
    \caption{Gemessene Widerstände und Spannung und die daraus berechnete Widerstands- und Spannungsdifferenz von \ce{Nd2O3}}
    \label{tab:messnd}
    \begin{tabular} {S[table-format=1.1] S[table-format=4.0] S[table-format=1.1] S[table-format=3.0] S[table-format=1.2] S[table-format=1.1]}
        \toprule
        {$U_\text{ohne} \mathbin{/} \si{\milli\volt}$} & {$R_\text{ohne} \mathbin{/} \si{\milli\ohm}$} & {$U_\text{mit} \mathbin{/} \si{\milli\volt}$}    &
        {$R_\text{mit} \mathbin{/} \si{\milli\ohm}$}   &  {$\symup{\Delta}R \mathbin{/} \si{\ohm}$}    & {$\symup{\Delta}U \mathbin{/} \si{\milli\volt}$}\\
    \midrule
    3.4 & 2200 & 3.6 & 1900 & 0.30 & 0.2 \\
    3.4 & 2150 & 3.5 & 1900 & 0.25 & 0.1 \\
    3.4 & 2200 & 3.6 & 1900 & 0.30 & 0.2 \\
    \bottomrule
\end{tabular}
\end{table}
Die mit der Widerstandsdifferenz ermittelten Suszeptibilität lässt sich mit der Gleichung \eqref{eqn:chir} bestimmen. 
Mit Hilfe der Gleichung \eqref{eqn:chil} kann die Suszeptibilität mit der Spannungsdifferenz berechnet werden, wobei die Speisespannung $U_\text{Sp} = \SI{1}{\volt}$ betrug.
Für die beiden Proben ergibt sich die Tabelle \ref{tab:expsusz} mit den Mittelwerten der Widerstands- und Spannugsdifferenz, die Suszeptibilitäten $\chi_R$ und
$\chi_U$.
\begin{table}
    \centering
    \caption{Mittelwerte der Widerstands- und Spannungsdifferenzen und die Suszeptibilitäten von $\ce{Dy2O3}$ und $\ce{Nd2O3}$}
    \label{tab:expsusz}
    \begin{tabular} {S[table-format=3.0] 
                     S[table-format=1.3] @{${}\pm{}$} S[table-format=1.3]
                     S[table-format=1.3] @{${}\pm{}$} S[table-format=1.3] 
                     S[table-format=1.2] @{${}\pm{}$} S[table-format=1.2]
                     S[table-format=2.2] @{${}\pm{}$} S[table-format=1.2]}
        \toprule
        {$\text{Probe}$} & 
        \multicolumn{2}{c}{$\bar{\symup{\Delta}R} \mathbin{/} \si{\ohm}$} & 
        \multicolumn{2}{c}{$\bar{\symup{\Delta}U} \mathbin{/} \si{\milli\volt}$} & 
        \multicolumn{2}{c}{$\chi_R \cdot \num{e-2}$} & 
        \multicolumn{2}{c}{$\chi_U \cdot \num{e-2}$} \\
    \midrule
    {$\ce{Dy2O3}$} & 1.66 & 0.05  & 4.20   & 0.15  & 2.42 & 0.07 & 12.2 & 4    \\
    {$\ce{Nd2O3}$} & 0.283& 0.017 & 0.167  & 0.033 & 0.61 & 0.04 & 0.71 & 0.14 \\
    \bottomrule
\end{tabular}
\end{table}
Nach der Berechnung des realen Querschnitts können die Suszeptibilitäten nach der Gleichung \eqref{eqn:chir} berechnet werden.
Der Widerstand $R_3$ beträgt $\SI{1}{\kilo\ohm}$. Der Spulenquerschnitt hat den Wert $F = \SI{86.6}{\milli\metre\squared}$
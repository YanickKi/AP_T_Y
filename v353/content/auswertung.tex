\section{Auswertung}
\label{sec:Auswertung}
Jegliche Fehlerrechnung wurde mit der Python-Bibliothek uncertainties \cite{uncertainties} absolviert.
Trotz dessen sind die Formeln für die Unsicherheiten in den jeweiligen Abschnitten angegeben.
Allgemeine Rechnungen wurden mit der Python-Bibliothek numpy \cite{numpy} automatisiert. \\
Die Spannung $U_0$ beträgt bei allen Durchführungen $\SI{6.4}{\volt}$
\subsection{Bestimmung der Zeitkonstante mit dem Entladevorgang}
Die während der Durchführung gemessenen Werte für $U_\text{C}$ in Abhängigkeit von der Zeit sind in Tabelle \ref{tab:UCt} aufgeführt.
\begin{table}
    \centering
    \caption{Gemessene Kondensatorspannung $U_\text{C} \left (t \right )$}
    \label{tab:UCt}
    \begin{tabular}{S[table-format=1.1] S[table-format = 1.1]}
        \toprule
        {$ t \mathbin{/} \si{\milli\second}$} & {$U_\text{C} \mathbin{/} \si{\volt}$} \\
        \midrule
        0   & 6.2 \\
        0.1 & 5.3 \\
        0.2 & 4.7 \\
        0.3 & 4.1 \\
        0.4 & 3.5 \\
        0.5 & 2.9 \\
        0.6 & 2.5 \\
        0.7 & 1.9 \\
        0.8 & 1.5 \\
        0.9 & 1.1 \\
        1   & 0.7 \\
        1.1 & 0.5 \\
        1.2 & 0.1 \\
        \bottomrule        
    \end{tabular}
\end{table}
\begin{figure}
        \centering
        \caption{Verlauf der Spannung bei einem Entladevorgang}
        \label{fig:discharge}
        \includegraphics{build/uct.pdf}
\end{figure}
\noindent Bei dem Spannungsverlauf in Abblidung \ref{fig:discharge} ist zunächst anzumerken, dass dort zwei Ausgleichsgeraden angelegt wurden. 
Dies liegt der Ursache zu Grunde, dass der letzte Punkt der Messreihe einen sehr starke Abweichung zeigt.
Die gesamte Ausgleichsgerade beschreibt die komplette Messreihe, wobei der letzte Messwert bei der kritischen Ausgleichsgerade ausgelassen wurde.
Im Folgenden werden die Rechungen mit den Parametern der kritischen Ausgleichsgeraden durchgeführt.
Für die Bestimmung der Zeitkonstante $RC$ wird Geleichung \eqref{eqn:Charge} verwendet.
Jedoch wird die Ladung $Q$ durch die Spannung $U_\text{C}$ ersetzt.
Umgestellt und logarithmiert ergibt sich 
\begin{equation}
    \ln \left( \frac{U_\text{C}}{U_0} \right ) =  - \frac{t}{RC} \; \text{.} 
\end{equation} 
Die Ausgleichsgerade hat die Form $ \ln \left( \sfrac{U_\text{C}}{U_0} \right ) = at + b$, wobei $a = - \sfrac{1}{RC}$ gilt.
Die Parameter haben die Werte 
\begin{align*}
    a &= \SI{-0.3447(229)}{\per\milli\second} \\
    b &= \num{0.3169(149)} \; \text{.}
\end{align*}
Der Wert für $a$ in $RC = - \sfrac{1}{a}$ eingesetzt ergibt
\begin{equation*}
    \tau = RC = \SI{2.90(19)}{\milli\second} \; \text{.}
\end{equation*}
\subsection{Frquenzabhängige Spannunsgmessung}
Die Messwerte für die Kondensatorspannung $U_\text{C}$ in Abhängigkeit von der Frequenz $\omega$ sind in Tabelle \ref{tab:ucw} aufgelistet.
\begin{table}
    \centering
    \caption{Gemessene Kondensatorspannung $U_\text{C} \left( \omega \right)$}
    \label{tab:ucw}
    \begin{tabular} {S[table-format = 4.0] S[table-format = 1.4]}
        \toprule
        {$\omega \mathbin{/} \si{\hertz}$} & {$U_\text{C} \mathbin{/} \si{\volt}$}\\
        \midrule
        10    & 6.4    \\
        20    & 6.4    \\
        40    & 6.2    \\
        60    & 6.0    \\
        80    & 5.6    \\
        100   & 5.4    \\
        200   & 3.6    \\
        400   & 2.0    \\
        600   & 1.4    \\
        800   & 1.05   \\
        1000  & 0.85   \\
        2000  & 0.44   \\
        4000  & 0.2    \\
        6000  & 0.014  \\
        8000  & 0.0105 \\
        10000 & 0.0085 \\
        \bottomrule
    \end{tabular}
\end{table}
Um die nichtlineare Ausgleichkurve anlegen zu können wird zunächst die Funktion der Amiplitude benötigt, welche in Gleichung \eqref{eqn:ampli} aufgführt ist. 
Fortlaufend wird zur Vereinfachung $R^2C^2$ zu $\tau^2$ umdefiniert.
Die Frequenz $\omega$ wird zur Klarifizierung in $x$ umbennant.
Somit nimmt die Funktion der Ausgleichskurve die Gestalt
\begin{equation}
    f(x) = \frac{U_\text{C}}{U_0} \left( x \right) = \frac{1}{\sqrt{1 + x^2 \tau^2}}
\end{equation}
an.
Mittels der Ausgleichsrechnung in Python ergibt sich $\tau$ zu 
\begin{equation*}
    \tau = \SI{-7.2(1)}{\milli\second}
\end{equation*}
Somit ergibt sich die Zeitkonstante $RC$ zu   
\begin{equation}
    RC =  \SI{7.2(1)}{\milli\second} \; \text{.}
\end{equation}
Die Messdaten samt Fit sind in Abblidung \ref{fig:ucw} visualisiert
\begin{figure}
    \centering
    \caption{Kondensatorspannung $U_\text{C} \left( \omega \right)$}
    \label{fig:ucw}
    \includegraphics{build/ucw.pdf}
\end{figure}
\subsection{Bestimmung der Zeitkonstante mittels frequenzabhängigen Phasenverschiebung}
Die Messwerte für die Frequenz $\omega$, der Zeitabstände der Nulldurchgänge $a$ und der Schwingundsdauern $b$ sind in Tabelle \ref{tab:ab} aufgetragen.
Zusätzlich beinhaltet diese Tabelle die aus den Messdaten berechneten Phasenverschiebungen, welche sich mittels  Gleichung \eqref{eqn:phiab} berechnen lassen.
\begin{table}
    \centering
    \caption{Gemessene Frequenz, Zeitabstände der Nulldurchgänge $a \left ( \omega \right )$ und Schwingundsdauern $b \left ( \omega \right )$}
    \label{tab:ab}
    \begin{tabular}{S[table-format = 4.0] S[table-format = 1.3] S[table-format = 3.3]}
        \toprule
        {$\omega \mathbin{/} \si{\hertz}$} & {$a \mathbin{/} \si{\milli\second}$} & {$b \mathbin{/} \si{\milli\second}$} \\
        \midrule
        10    & 1.2   & 110 \\
        20    & 1.5   & 55  \\
        40    & 1.7   & 25  \\
        60    & 1.15  & 16.6\\
        80    & 1.1   & 12.6\\
        100   & 1.08  & 10\\
        200   & 0.8   & 5\\
        400   & 0.52  & 2.5\\
        600   & 0.36  & 1.7\\
        800   & 0.28  & 1.24\\
        1000  & 0.23  & 1\\
        2000  & 0.12  & 0.49\\
        4000  & 0.06  & 0.25\\
        6000  & 0.04  & 0.168\\
        8000  & 0.031 & 0.128\\
        10000 & 0.024 & 0.1 \\
        \bottomrule
    \end{tabular}
\end{table}
Um die Zeitkonstante mit Hilfe der Phasenverschiebung graphisch zu bestimmen, wird die Gleichung \eqref{eqn:ficken} benötigt.
Erneut wird die Zeitkonstante $RC$ zu $\tau$ umdefiniert und die Frequenz $\omega$ in x unbenannt.
Somit erhält die Ausgleichkurve die Darstellung 
\begin{equation}
    f \left( \omega \right) = phi \left ( \omega \right ) = \arctan 
\end{equation}
\begin{figure}
    \centering
    \caption{Phasenverschiebung $\varphi \left ( \omega \right )$}
    \label{fig:phiw}
    \includegraphics{build/phasefreq.pdf}
\end{figure}
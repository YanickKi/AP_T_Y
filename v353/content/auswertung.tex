\section{Auswertung}
\label{sec:Auswertung}
Jegliche Fehlerrechnung wurde mit der Python-Bibliothek uncertainties \cite{uncertainties} absolviert.
Trotz dessen sind die Formeln für die Unsicherheiten in den jeweiligen Abschnitten angegeben.
Allgemeine Rechnungen wurden mit der Python-Bibliothek numpy \cite{numpy} automatisiert.
\begin{figure}
        \centering
        \caption{Verlauf der Spannung bei einem Entladevorgang}
        \label{fig:discharge}
        \includegraphics{build/uct.pdf}
\end{figure}
Bei dem Spannungsverlauf ist zunächst anzumerken, dass dort zwei Ausgleichsgeraden angelegt wurden. 
Dies liegt der Ursache zu Grunde, dass der letzte Punkt der Messreihe einen sehr starke Abweichung zeigt.
Die gesamte Ausgleichsgerade beschreibt die komplette Messreihe, wobei der letzte Messwert bei der kritischen Ausgleichsgerade ausgelassen wurde.
Im Folgenden werden die Rechungen mit den Parametern der kritischen Ausgleichsgeraden durchgeführt.
Für die Bestimmung der Zeitkonstante $RC$ wird Geleichung \eqref{eqn:Charge} verwendet.
Jedoch wird die Ladung $Q$ durch die Spannung $U$ ersetzt.
Umgestellt und logarithmiert ergibt sich 
\begin{equation}
    \ln \left( \frac{U}{U_0} \right ) =  - \frac{t}{RC} \; \text{.} 
\end{equation} 
Die Ausgleichsgerade hat die Form $ \ln \left( \sfrac{U}{U_0} \right ) = at + b$, wobei $a = \sfrac{1}{RC}$ gilt.
Die Steigung der Ausgleichsgerade und somit der Betrag der Zeitkonstanten ergibt sich mittels linearer Regression zu 
\begin{equation}
    \tau = \abs*{RC} = \SI{2.2058(1465)}{\milli\second} \; \text{.}
\end{equation}

\section{Diskussion}
\label{sec:Diskussion}
Die in dem Abschnitt \ref{sec:frank} bestimmte Anregungsenergie von $\ce{Hg}$-Atomen weicht von dem Literaturwert\cite{lit} um
\begin{equation*}
   \left | \frac{E_\text{lit} - E_\text{exp}}{E_\text{lit}} \right | = \left | \frac{\SI{4.9}{\electronvolt}-\SI{5}{\electronvolt} }{\SI{4.9}{\electronvolt}} \right |
    = \SI{2.04}{\percent}
\end{equation*} 
ab, woraus sich schließen lässt, dass die Energie mit einer hohen Präzision bestimmt werden konnte.
Trotz der wenigen Messwerte ließ sich eine geringe Abweichung erzielen, was an der Tatsache, dass beinahe nur Extrema und somit nur die für die 
Auswertung wichtigen Messdaten gemssen wurde, liegen könnte.
Jedoch ist anzumerken, dass die Messung dadurch erschwert worden, dass das Amperemeter ohne vorhandene Beschleunigungsspannung einen stetig ansteigenden 
Strom anzeigte, was in eine (wenn auch kleine) systematische Abweichung gemündet sein könnte.
Des weiteren wurde die Beschleunigungsspannung bei der Aufnahme der Frank-Hertz-Kurve nur bis $\SI{30}{\volt}$ aufgenommen, da der Auffängerstrom bei 
weiterer Erhöhung der Beschleunigungsspannung nur stetig anstieg und nicht mehr abfiel. 

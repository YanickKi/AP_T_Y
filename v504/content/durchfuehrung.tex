\section{Durchführung}
\label{sec:Durchführung}
\subsection{Kennlinien und Sättigungsstrom}
Zur Bestimmung der Kennlinien wurde ein Aufbau wie in Abbildung \ref{fig:aufbaua} benutzt.
\begin{figure}
    \centering
    \caption{Schaltung zur Aufnahme der Diodenkennlinien \cite{v504}}
    \label{fig:aufbaua}
    \includegraphics[width = 0.5 \textwidth]{pics/Aufbaua.png}
\end{figure}
Die Kennlinien wurden durch variiation der Spannung zwischen Anode und Kathode entnommen. Dabei wurde für jede neue Kennlinie der Heizstrom um $\SI{0.1}{\ampere}$ erhöht.
Die dazugehörigen Messwerte und Heizspannungen sind in Tabelle \ref{tab:charcurve} aufgeführt. 
An Stelle eines XY-Schreibers wurde an einem Amperemeter der Diodenstrom $I_\text{A}$ abgelesen.
Aus den Messwerten lässt sich auch auf den Sättigungsstrom schließen.

\subsection{Anlaufstrom und Austrittsarbeit}
Die Messaperatur wird wie in Abbildung \ref{fig:aufbauc} aufgebaut.
\begin{figure}
    \centering
    \caption{Schaltung zur Aufnahme des Anlaufstroms \cite{v504}}
    \label{fig:aufbauc}
    \includegraphics[width = 0.5 \textwidth]{pics/Aufbauc.png}
\end{figure}
Wesentliche Unterschiede zu dem Aufbau \ref{fig:aufbaua} sind dabei nur die Umpolung Spannungsgerätes und dass Austauschen des Amperemeters mit einem empfindlicheren $\si{\nano\ampere}$ Amperemeter. Durch den eingebauten Verstärker ist dieses Gerät jedoch Störungsanfälliger.
Bei der Messung wird der Heizstrom auf \SI{2.5}{\ampere} eingestellt und die Gegenspannung langsam hochskaliert. Die Messwerte werden an dem Amperemeter mit zugehöriger Gegenspannungen abgelesen.

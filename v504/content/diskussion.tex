\section{Diskussion}
\label{sec:Diskussion}
Eine zu beachtende errechnete Größe ist die Austrittsarbeit. Denn diese weicht vom Literaturwert\cite{lit}
\begin{equation*}
    e_0\phi_\text{Lit} =  \SI{4.54}{\electronvolt} - \SI{4.6}{\electronvolt}
\end{equation*}
nicht stark ab. Das Verhältnis des errechneten Wertes und des Literaturwerts liegt bei
\begin{equation*}
    \eta = \frac{\SI{4.612}{\electronvolt}}{\SI{4.57}{\electronvolt}} \approx \SI{101.58}{\percent} \; \text{,}
\end{equation*}
was einer Abweichung von ungefähr $\SI{1.58}{\percent}$ entspricht.
Somit lässt sich sagen, dass die Austrittsarbeit von Wolfram relativ genau bestimmt werden konnte.
Des Weiteren sind die beiden stark abweichenden Punkte in der Abbildung \ref{fig:exponent} auffällig. 
Diese beiden Ausreißer enstanden durch die Korrektur des Spannungsabfalls über das Nanoamperemeter.
Eine mögliche Fehlerquelle könnte eine falsche Skalierung des Nanoamperemeters sein. 
Diese fließen wegen der Korrektur \eqref{eqn:Korrektur} stark in die Berechung ein.
Ein weiterer Grund könnte eine Störung bei dem Messvorgang an dem Nanoamperemeter sein, da dieses sehr empfindlich auf Bewegungen oder 
benachbarte Objekte reagierte.
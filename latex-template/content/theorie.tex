\section{Theorie}
\label{sec:Theorie}
Ohne äußere Einflüsse ist die Richtung des Wärmeflusses immer vom warmen Reservoir in das kalte Reservoir.
Durch anwenden von mechanischer Arbeit $A$ lässt sich dieser Prozess auch in die andere Richtung durchführen.
\subsection{Das Prinzip der Wärmepumpe}
Nach dem ersten Hauptsatz der Thermodynamik beträgt die an das wärmere Medium abgegebene Wärmemenge $Q_1$, die summe der vom kühleren Medium aufgenommenen
Wärmeenergie $Q_2$ und der Arbeit $A$. Also gilt
\begin{equation}
    Q_1=Q_2+A
    \label{eqn:H1}
\end{equation}

Die Güteziffer $\nu$ der Wärmepumpe beschreibt das Verhältnis der abgegebenen
Wärmemenge $Q_1$ zur aufgewandten Arbeit $A$

\begin{equation}
    \nu=\frac{Q_1}{A}
    \label{eqn:Güte}
\end{equation}

Der zweite Hauptsatz der Thermodynamik führt für die reduzierten Wärmemengen 
zu der Beziehung, dass deren Summe $\int \frac{\symup{d}Q}{T}$ null beträgt. 
Aus dieser folgt

\begin{equation}
    \frac{Q_1}{T_1} - \frac{Q_2}{T_2} = 0
    \label{eqn:redWärm}
\end{equation}

Jedoch gilt dies nur bei Idealen reversiblen Prozessen. In der technischen Anwendung gilt

\begin{equation}
    \frac{Q_1}{T_1} - \frac{Q_2}{T_2} > 0
    \label{eqn:ungWärme}
\end{equation}

aus \ref{eqn:H1} und \ref{eqn:redWärm} folgt

\begin{equation*}
    Q_1 = A + \frac{T_2}{T_1} \cdot Q_1 \; \text{,}
\end{equation*}

aus der Formel für die Güteziffer\ref{eqn:Güte}, für einen reversiblen Vorgang, folgt die Güteziffer einer Idealen Wärmepumpe
\begin{equation}
    \nu_\text{id} = \frac{Q_1}{A} = \frac{T_1}{T_1 - T_2}
    \label{eqn:Id}
\end{equation}

Für die reale Wärmepumpe folgt aus \ref{eqn:H1} und \ref{eqn:ungWärme}

\begin{equation}
    \nu_\text{real} < \frac{T_1}{T_1 - T_2} \; \text{.}
    \label{eqn:Re}
\end{equation}

Aus den Gleichungen \ref{eqn:Id} und \ref{eqn:Re} folgt, dass eine Wärmepumpe effizienter Arbeitet, je kleiner die Temperaturdifferenz
zwischen den beiden Wärmereservoirs ist. Der Vorteil einer Wärmepumpe liegt außerdem in der möglichkeit eine Wärmemenge $Q_2$, welche frei Verfügbar ist 
auszunutzen um Preisgünstig $Q_1$ zu heizen. Damit kann die verrichtete Arbeit, unter günstigen Bedingungen erheblich kleiner sein als die gewonnene Wärmemenge $Q_1$.
Damit folgt das eine Wärmepumpe gegenüber einem Wärmegewinnungsverfahren, welches machanische Energie direkt in Wärme umwandelt, effizienter ist.
Hier ist die erhaltene Wärmemenge höchstens gleich der mechanischen Arbeit, also
\begin{equation*}
    Q_{1_\text{direkt}}\leq A
\end{equation*}

im Gegensatz zu

\begin{equation*}
    Q_{1_\text{rev}}=A\,\frac{T_1}{T_1-T_2}
\end{equation*}.
\subsection{Die Arbeitsweise einer Wärmepumpe}
\begin{figure}
    \centering
    \includegraphics[scale=0.4]{aufbau.pdf}
    \caption{Aufbau einer Wärmepumpe [1]}
    \label{fig:aufbau2}
\end{figure}
\section{Auswertung}
\label{sec:Auswertung}
\subsection{Graphische Visualisierung}
\begin{figure}
  \centering
  \includegraphics{temperatur.pdf}
  \caption{Temperaturverlauf beider Reservoirs.}
  \label{fig:temperatur}
\end{figure}
\subsection{Nicht-lineare Ausgleichsrechnung}
Der nicht-lineare Zusammenhang zwischen der Zeit und der Temperatur lässt sich mittels einer quadtratischen Ausgleichskurve, welche durch 
\begin{equation}
  T(t) = At^2 + Bt + C
\end{equation}
beschrieben wird, approximieren. Mittels Rechnungen in Python ergaben sich die Ausgleichskfunktionen zu
\begin{align}
  T_1(t) &= - 0.0116t^2 + 1.2168 t + 294.7008 \\
  T_2(t) &= \phantom{-}  0.0034 t^2 - 0.6725 t + 295.8702
\end{align}
\subsection{Differentialquotient}
Die Differentialquotienten von $T_1(t)$ und $T_2(t)$ sind durch
\begin{align}
  \frac{\symup{d} T_1(t)}{\symup{d}t} = - 0.0232t + 1.2168  \\
  \frac{\symup{d} T_2(t)}{\symup{d}t} = \phantom{-} 0.0068 t - 0.6725 
\end{align}
gegeben. Die in die Differentialquotienten eingesetzen Zeitpunkte $t = 10$, $t = 15$, $t = 20$ und $t = 35$ ergeben folgende Tabelle:
\begin{table}
  \centering
  \caption{Ergebnisse der Differentialquotienten}
  \label{tab:Differentialquotient}
  \sisetup{table-format = 1.4}
  \begin{tabular}{S[table-format = 2.0] S S}
    \toprule
    {$t \mathbin{/} \si{\second}$} & {$\frac{\symup{d} T_1(t)}{\symup{d}t} \mathbin{/} \si{\kelvin\second\tothe{-1}}$} & 
    {$\frac{\symup{d} T_2(t)}{\symup{d}t} \mathbin{/} \si{\kelvin\second\tothe{-1}}$} \\
    \midrule
    600  & 0.0164 &-0.0101 \\
    900  & 0,0145 &-0.0095 \\
    1200 & 0.0125 &-0.0089 \\
    2400 & 0.0067 &-0.0072 \\
    \bottomrule
  \end{tabular}
\end{table}
\subsection{Güteziffer}
Um die reale Güteziffer zu bestimmen dient die Gleichung \eqref{eqn:Güteziffer}.
Für die Wärmekapazität des Wassers gilt $m_1 c_\text{w}(T) = \SI{4}{\kilo\gram}\cdot c_\text{w}(T)$,
für die der Kupferschlangen $m_\text{k}c_\text{k} = \SI{750} {\joule\per\kelvin}$. Um die Genauigkeit der Rechnung zu erhöhen wurde hier die Temperaturabhängigkeit
der spezifischen Wärmekapazität $c_\text{w}(T)$ von Wasser berücksichtigt. Für die Berechnung der idealen Güterziffer $\nu_\text{id}$ ist die Beziehung \eqref{eqn:Id} von Nutzen.
Mittels Rechnungen ergaben sich die Werte der Güterziffern zu:
\begin{table}
  \centering
  \caption{Vergleich $\nu_\text{re}$ zu $\nu_\text{id}$}
  \label{tab:Gueterziffer}
  \sisetup{table-format = 2.4}
  \begin{tabular}{S[table-format = 3.2] S[table-format = 1.4] S[table-format = 1.4] S S}
    \toprule
    {$T_1 \mathbin{/} \si{\kelvin}$} & {$ c_\text{w}(T_1) \mathbin{/} \si{\kilo\joule\kilo\tothe{-1}\gram\tothe{-1}\kelvin\tothe{-1}}$} & 
    {$\nu_\text{re}$} & {$\nu_\text{id}$} & {$\frac{\nu_\text{re}} {\nu_\text{id}} \si{\percent}$} \\
    \midrule
    306.05 & 4.1783 & 2.3866 & 18.3263 & 13.0230\\
    310.75 & 4.1783 & 2.1101 & 12.6321 & 16.7045\\    
    314.55 & 4.1787 & 1.8193 & 10.1468 & 17.9293\\  
    323.45 & 4.1807 & 0.9596 &  6.8238 & 14.0621\\    
    \bottomrule                                       
  \end{tabular}                                     
\end{table}
\subsection{Massendurchsatz}
Der Massendurchsatz lässt sich mittels Gleichung \eqref{eqn:Massendurchsatz} ermitteln, jedoch fehlt zunächst der Wert der Verdampfungswärme L, welche 
mit Hilfe einer Dampfdruck-Kurve bestimmt werden kann. Diese Kurve wird durch 
\begin{equation}
  \ln (p) = -\frac{L}{RT} + const \label{eqn:Verdampfungswaerme}
\end{equation}
beschrieben, wobei R die allgemeine Gaskonstante ist. Zwecks der Definition $a \coloneq -\frac{L}{R}$ erhält man die Vereinfachung
\begin{equation}
  \ln (p) = \frac{a}{T} + const
\end{equation}
\begin{figure}
  \centering
  \includegraphics{Verdampfungswaerme.pdf}
  \caption{Dampfdruck-Kurve}
  \label{fig:Dampfdruck}
\end{figure}
Hierbei ist zu beachten, dass das Argument des natürlichen Logarithmus nicht einheitslos ist (was normalerweise
der Fall sein müsste), da die Gleichung \eqref{eqn:Verdampfungswaerme} durch Vereinfachungen angenähert wurde.
Nach der Durchführung der linearen Regression, ergeben sich $a = -2462.4863$ und die Konstante zu $10.1354$.
Rücksubstituieren von $a = -\frac{L}{R}$ und umstellen nach $L$ führt zu $ L = - Ra$. Einsetzen von $a$ und der allgemeinen Gaskonstante 
$R = \SI{8.314472}{\joule\per\mole\per\kelvin}$ liefert
\begin{equation}
  L = \SI{20474.2733}{\joule\per\mol}\, .
\end{equation}
Um die Einheit von $\si{\joule\per\mole}$ auf Energie pro Masse, also $\si{\joule\per\gram}$ umzurechnen, dividiert man durch die Molare
Masse von Dichlordifluormethan ($M = \SI{120.91}{\gram\per\mole}$), sodass man letzendlich
\begin{equation}
  L = \SI{106.5588}{\joule\per\gram}
\end{equation}
erhält.
Nach der Bestimmung der Verdampfungswärme folgt die Bestimmung des Massendurchsatzes,
wobei in diesem Fall die Temperaturabhängigkeit der spezifischen Wärmekapazität erneut berücksichtigt wird.
\begin{table}
  \centering
  \caption{Errechneter Massendurchsatz}
  \label{tab:Massendurchsatz}
  \sisetup{table-format = 1.4}
  \begin{tabular}{S[table-format = 3.2] S S S}
    \toprule
    {$T_2 \mathbin{/} \si{\kelvin}$} & {$ c_\text{w}(T_2) \mathbin{/} \si{\kilo\joule\kilo\tothe{-1}\gram\tothe{-1}\kelvin\tothe{-1}}$} & 
    {$\frac{\symup{d} T_2(t)}{\symup{d}t} \mathbin{/} \si{\kelvin\second\tothe{-1}}$} & {$\frac{\symup{d}m}{\symup{d}t} \mathbin{/} \si{\gram\second\tothe{-1}}$} \\
    \midrule
    289.35 & 4.1849 & -0.0101 &-1.6577\\
    286.15 & 4.1880 & -0.0095 &-1.5603\\    
    283.55 & 4.1922 & -0.0089 &-1.4632\\  
    276.05 & 4.2077 & -0.0072 &-1.1879\\    
    \bottomrule                                       
  \end{tabular}                                     
\end{table}\\
\subsection{Mechanische Kompressorleistung}
\begin{table}
  \centering
  \caption{Errechnete mechanische Kompressorleistung}
  \label{tab:Mechanische Kompressorleistung}
  \begin{tabular}{S[table-format = 3.2] S[table-format = 2.0] S[table-format = 1.0] S[table-format = 2.4] S[table-format = 2.4]}
    \toprule
    {$T_2 \mathbin{/} \si{\kelvin}$} & {$p_1 \mathbin{/} \si{bar}$} & {$p_2 \mathbin{/} \si{bar}$}
    & {$\rho \mathbin{/} \si{\gram\liter\tothe{-3}}$} & {$N_\text{mech} \mathbin{/} \si{W}$}\\
    \midrule
    289.35 & 8.0  & 4.0 & 20.8060 & 20.2264\\
    286.15 & 9.0  & 3.6 & 18.9348 & 25.2380\\    
    283.55 & 10.0 & 3.4 & 18.0469 & 27.8938\\  
    276.05 & 12.0 & 3.2 & 17.4468 & 27.4276\\    
    \bottomrule                                       
  \end{tabular}                                     
\end{table}
\subsection{Gründe für die vergleichsweise schlechte Güteziffer}
Während des Vergleichens beider Güteziffern wird auffällig, dass das Verhältnis sehr niedrig, im Bereich von $10$ bis $\SI{20}{\percent}$, liegt.
In der Gleichung \eqref{eqn:redWärm} wird gefordert, dass der Prozess der Wärmeübertragung ein reversibler Prozess ist,
und somit jede Energie, welche kurzfristig verloren geht, zurückgewonnen werden kann. Doch in der Realität ist diese nicht mehr 
zurück zu gewinnen. Beispielsweise kann die Energie in Form von Wärmeenergie an die Umwelt abgegeben werden.
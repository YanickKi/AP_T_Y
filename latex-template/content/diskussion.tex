\section{Diskussion}
\label{sec:Diskussion}
Während des Vergleichens beider Güteziffern \eqref{tab:Gueterziffer} wird auffällig, dass das Verhältnis sehr niedrig (im Bereich von $10$ bis $\SI{20}{\percent}$), ist.
In der Gleichung \eqref{eqn:redWärm} wird gefordert, dass der Prozess der Wärmeübertragung ein reversibler Prozess ist,
und somit jede Energie, welche kurzfristig verloren geht, zurückgewonnen werden kann. Doch in der Realität ist diese nicht mehr zu $\SI{100}{\percent}$
zurück zu gewinnen. Beispielsweise kann die Energie in Form von Wärmeenergie an die Umwelt abgegeben werden.
Außerdem wird bei Betrachtung der berechneten Werte zur zeit $t = \SI{2100}{\second}$ auffällig, dass diese sehr stark von den der anderen Zeiten abweicht. Dies liegt
der Ursache zu Grunde, dass der Zeitabstand größer als der der vorigen Zeitpunkte ist. Abschließend weicht der letze Punkt (bei $t = \SI{0}{\second}$)
in der Abbildung \eqref{fig:Dampfdruck} stark von der Regressionskurve ab. Wenn man nun in Daten der Tabelle schaut, erkennt man, dass die Differenz
zu dem Druck des nächsten Zeitpunktes, im Vergleich zu den anderen Zeitintevallen, sehr groß ist. Somit kann man feststellen, dass die starke Abweichung entweder
einer unsauberen Messung oder einem plötzlichen Druckanstieg zu Grunde liegt.
\newpage
\section{Diskussion}
\label{sec:Diskussion}
In der Tabelle \ref{tab:wavelengths} wird auffällig, die Impulse des dritten und fünften Messvorgangs eine starke Abweichung vorweisen.
die Abweichung des dritten Messwerts könnte an der Tatsache liegen, dass die Mikrometerschraube vor dem Messvorgang nicht auf $\SI{0.0}{\milli\metre}$
zurückgesetzt wurde. 
Diese Vermutung wird durch die Tatsache gestärkt, dass die alle anderen Messvorgänge bei  $\SI{0.0}{\milli\metre}$ gestartet wurden.
Die Abweichung des fünften Werts könnte durch eine andere Motoreinstellung enstanden sein, denn in diesem Messdurchgang wurde der Motorgeschwindigkeit von Stufe
eins auf Stufe zwei gestellt.
Nach Berechnung des Verhältnisses des angegeben Wertes der Wellenlänge der Lichtquelle von $\lambda_\text{An}$ und des errechneten Wertes 
\begin{equation*}
    \frac{\lambda}{\lambda_\text{An}} = \frac{\SI{658.14}{\nano\metre}}{\SI{635}{\nano\metre}} = \SI{103.64}{\percent} \; \text{,}
\end{equation*}
was einer Abweichung von $\SI{3.64}{\percent}$ entspricht, lässt sich sagen, dass die Auslassung zweier Messwerte die Qualität der Messung nicht 
stark beeinträchtigt.
Nach Betrachtung der Tabelle \ref{tab:fractionindex} ist zu erkennen, dass alle zweite gemessenen Impulse sehr klein sind. 
Diese wenigen Impulse wurden während des Auffülen mit Luft gemessen. 
Bei der Evakuuierung wurden die vielen Impulse gemessen, so dass sich dort ein Zusammenhang erkennen lässt.
Das Verhältnis des Literaturwertes\cite{fractionindex} und des berechneten Wertes liegt bei 
\begin{equation*}
    \frac{n}{n_\text{Lit}} = \frac{\num{1.000367}}{\num{1.000292}} = \SI{100.0075}{\percent} \; \text{.}
\end{equation*}
Somit ergibt sich eine Abweichung  von $\SI{0.0075}{\percent}$. 
Dadurch lässt sich sagen, dass die Messung trotz den Schwankungen in den Messungen ziemlich genau ist.
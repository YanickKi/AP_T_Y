\section{Auswertung}
\label{sec:Auswertung}
\subsection{Geiger-Müller Charakteristik}
Zur Bestimmung der Charakteristik des Zählors wurde ein Spannung mit Schritten von $\symup{\Delta} U = \SI{10}{\volt}$ angelegt.
Dazu wurde die Teilchenanzahl pro Zeitintervall $N$ gemessen. 
Das Zeitintervall beträgt hierbei $\symup{\Delta} t = \SI{60}{\second}$.
Zur Ausgleichsrechung des Plateaus wird das Intervall von $\SI{370}{\volt}$ bis $\SI{630}{\volt}$ gewählt.
Mittels Rechnungen in python lassen sich die Parameter der Ausgleichsgerade 
\begin{equation}
  N = aU + b
\end{equation}
zu 
\begin{align*}
  a &= \SI{1.1378(2407)}{\per\minute\per\volt} = ( \num{13.78(2407)} )  \,\frac{\si{\percent}}{100 \, \si{\minute\volt}} \\
  b &= \SI{9590.7346(1218237)}{\per\minute}
\end{align*}
bestimmen.
\begin{figure}
  \centering
  \caption{Charakteristik des Halogenzählrohrs}
  \includegraphics{build/char.pdf}
\end{figure}
Da Die Zählrate Poisson verteilt sind, lässt sich die Unsicherheit mit 
\begin{equation}
  \symup{\Delta} N = \sqrt{N}
\end{equation}
ermitteln.
\subsection{Bestimmung der Totzeit}
In der Tabelle \ref{tab:Zaehlrate} sind die gemessen Impulse pro $120$ Sekunden $N$ aufgetragen, wobei sich $N_1$ auf die gemessenen Impulse der ersten Quelle und $N_2$
auf die gemessenen Impulse der zweiten Quelle bezieht. $N_\text{1+2}$ beschreibt die Impulse, welche von beiden Quellen gleichzeitig ausgestrahlt werden.
\begin{table}
  \centering
  \caption{Zählraten der beiden Quellen}
  \label{tab:Zaehlrate}
  \begin{tabular}{S[table-format = 5] S[table-format = 5] S[table-format = 6]}
    \toprule
    {$N_1 \mathbin{/} (120 \si{\second})^{-1}$} & {$N_2 \mathbin{/} (120 \si{\second})^{-1} $} & {$N_\text{1+2} \mathbin{/} (120 \si{\second})^{-1}$} \\
    \midrule
    96041   & 76518   & 158479 \\
    \bottomrule
  \end{tabular}
\end{table}
Mittels Gleichung $REFERENZ$ lässt sich die Totzeit zu 
\begin{equation*}
  T \approx \SI{110(50)}{\micro\second}     
\end{equation*}
bestimmen.
Die Unsicherheit der Totzeit wird mit Hilfe der Gaußschen Fehlerfortpflanzung berechnet 
\begin{equation}
  \text{FEHLERFORMEL}
\end{equation}
An dem Oszilloskop kann eine Totzeit von 
\begin{equation*}
  T \approx \SI{160}{\micro\second}
\end{equation*}
abgelesen werden.
\begin{figure}
  \centering
  \caption{Freigesetze Teilchen pro eingefallende Teilchen}
  \label{Z}
  \includegraphics{build/cur.pdf}
\end{figure}
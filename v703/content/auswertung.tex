\section{Auswertung}
\label{sec:Auswertung}
Zur Bestimmung der Charakteristik des Zählors wurde ein Spannung mit Schritten von $\symup{\Delta} U = \SI{10}{\volt}$ angelegt.
Dazu wurde die Teilchenanzahl pro Zeitintervall $N$ gemessen. 
Das Zeitintervall beträgt hierbei $\symup{\Delta} t = \SI{60}{\second}$.
Zur Ausgleichsrechung des Plateaus wird das Intervall von $\SI{370}{\volt}$ bis $\SI{630}{\volt}$ gewählt.
Mittels Rechnungen in python lassen sich die Parameter der Ausgleichsgerade 
\begin{equation}
  N = aU + b
\end{equation}
zu 
\begin{align}
  a &= \SI{1.1378(2407)}{\per\minute\per\volt} \\
  b &= \SI{9590.7346(1218237)}{\per\minute}
\end{align}
bestimmen.
\begin{figure}
  \centering
  \caption{Charakteristik des Halogenzählrohrs}
  \includegraphics{build/char.pdf}
\end{figure}
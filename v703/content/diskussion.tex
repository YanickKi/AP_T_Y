\section{Diskussion}
\label{sec:Diskussion}
Zunächst wird auffällig, dass das Verhältnis aus der mit dem Oszilloskop bestimmten und der errechneten Totzeit klein ist, denn
das Verhältnis beträgt
\begin{equation*}
    \frac{T_\text{berechnet}}{T_\text{Oszilloskop}} = \SI{68.75}{\percent} \; \text{.}
\end{equation*}
Möglicherweise ist die die Platue-Länge falsch bestimmt worden. 
Die Beobachtung, dass die Platue-Steigung größer als eins ist ($\approx \SI{1.13}{\per\minute\per\volt}$), würde dies die Vermutung stützen. 
Jedoch würde dort das Gegenargument gelten, dass die Plateu-Länge nur sehr schwierig zu bestimmen ist, da der Graph in Abbildung \ref{fig:char}
schweren Schwankungen unterliegt. 
Somit ist der Beginn des Plateus nicht ganz eindeutig. \\
Im Hinblick auf die Problemtik könnte die Ursache auch bei dem Oszilloskop liegen. 
Die dortige Visualisierung zeugt von geringer Genauigkeit, dennd die einzelnen Kurven sind nicht komplett gebündelt sondern weichen von einander ab.
Somit liegt die Vermutung nahe, dass die optimalen Ablesepunkte zur Bestimmung der Totzeit woanders liegen und somit eine große Abweichung zu Stande kam.